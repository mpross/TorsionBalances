\documentclass{book}

\usepackage{amsmath}
\usepackage{amsfonts}
\usepackage{graphicx}
%\usepackage[paperwidth=5.125in, paperheight=8.25in]{geometry}

\begin{document}

\title{Torsion Balances:\\An Experimenter's Handbook}
\author{M.P.Ross for the E\"ot-Wash Group \\ University of Washington}

\maketitle

\tableofcontents

\setcounter{chapter}{-1}
\chapter{History}
\chapter{Introduction}
\section{Simple Torsion Balance}\label{simple}

\quad A torsion balance, in its simplest incarnation, is just an extended body, called the "pendulum," suspended from a thin wire, the "torsion fiber." This forms a rotational spring-mass system which has two intrinsic parameter (ignoring loss terms): the moment of inertia, $I$, and the torsional spring constant, $\kappa$. The primary degree of freedom of this system is rotation of the pendulum around the axis of the fiber which we call torsion. See Section~\ref{swing} for discussion of other degrees of freedom.

Restricting ourselves to only the torsional degree of freedom gives us:
\begin{equation}
I \ddot{\theta}(t)=\sum_i \tau_i(t)
\end{equation}
where $\theta$ is the angle of the pendulum about vertical, $t$ is time, and $\tau_i$ are the torques acting on the pendulum. Here we use Newton's notation for derivatives with respect to time, $\dot{x}=\partial x/\partial t$ and $\ddot{x}=\partial^2 x/\partial t^2$. 

\begin{figure}[!h]
\begin{centering}
\includegraphics[width=0.65\textwidth]{SimpleTorsionBalance.pdf}
\caption{A simple torsion balance system.}\label{simpleFig}
\end{centering}
\end{figure}

From Hooke's law, the torsional spring adds a restoring torque that follows:
\begin{equation}
\tau_{\text{spring}}(t) = -\kappa (\theta(t)-\theta_0)
\end{equation}
where $\theta_0$ is the equilibrium angle of the torsion balance. For most situations this is the dominant torque acting on the pendulum.

Due to historical reasons, the classical example of a torsion balance is a dumb-bell shaped pendulum suspended from a thin metal fiber, shown in Figure~\ref{simpleFig}. The pendulum is formed by a massless rod with two equal mass "test masses" attached to each end. The torsion fiber is then attached to the rod at equal distance to each test mass. This provides a prototypical model of a torsion balance which we will analyze in detail. Modern torsional balance apparatus typically have pendulums with more complex geometry and may have multiple suspension stages. 

\section{Loss Terms}

\quad There are two primary sources of loss in torsion balances: external and internal damping. External damping, sometimes called velocity damping, is caused by external forces acting on the pendulum that are proportional to the angular velocity of the pendulum, such as air friction. Internal damping, on the other hand, is caused by energy dissipation internal to the torsion fiber.

External damping adds a torque on the pendulum that is proportional to the angular velocity of the pendulum:
\begin{equation}
\tau_{\text{vel}}(t) = -\gamma \dot{\theta}(t)
\end{equation}
where $\gamma$ is the damping constant. Where as internal damping can be modeled by a complex spring constant:
\begin{equation}
\tau_{\text{spring}}(t) =  -\kappa(1+i\delta) \big(\theta(t)-\theta_0\big)
\end{equation}
where $\delta$ is the dimensionless internal loss parameter.

For most modern torsion balances, external damping is engineered away and is thus much smaller than the internal damping. Thus we will shelve discussion of external damping until Section \ref{gas}

\section{Equations of Motion}

\quad As mentioned in Section~\ref{simple}, the simple torsion balance is described with two primary parameters: the moment of inertia, $I$, which is determined by the pendulum geometry, and the torsional spring constant, $\kappa$, which is determined by the torsion fiber size and material. Adding in internal damping, this system obeys the following equation of motion:
\begin{equation}
I~\ddot{\theta}(t)+\kappa(1+i\delta)  \big(\theta(t)-\theta_0\big) = \tau_{\text{ext}}(t) \label{eom}
\end{equation}
where $\tau_{\text{ext}}(t)$ is the sum of all exterior torques acting on the pendulum.

If we assume a harmonic solution, $\theta(t)=A~e^{i\omega t}$, we can transform Equation~\ref{eom} into the Fourier domain to yield:
\begin{equation}
\big(-I\omega^2+\kappa(1+i\delta) \big) \tilde{\theta}(\omega)= \tau_{\text{ext}}(\omega) \label{four}
\end{equation}
It is convenient to define two parameters here: the resonant frequency, $\omega_0=\sqrt{\kappa/I}$, and the quality factor, $Q=\frac{1}{\delta}$. Equation \ref{four} can then be rearranged to:

\begin{equation}
 \tilde{\theta}(\omega)= \frac{\tau_{\text{ext}}(\omega)}{\kappa}\frac{1}{1-\frac{\omega^2}{\omega_0^2} +i/Q} \label{four2}
\end{equation}

\section{Response Function}\label{resp}

\quad The second dimensionless factor in Equation \ref{four2} is traditionally called the response function or transfer function of the system. It controls the amount of angle the pendulum gets for a given torque as a function of frequency. 

\begin{equation}
R(\omega)= \frac{1}{1-\frac{\omega^2}{\omega_0^2} +i/Q} \label{four3}
\end{equation}

For a pendulum with no damping, $Q\rightarrow\infty$, the response function has three distinct features. Below the resonant frequency, $\omega << \omega_0$, the response goes to unity. Above the resonant frequency, $\omega >> \omega_0$, the response function follows $R(\omega)=-\omega_0^2/\omega^2$ causing effect of high frequency torques to decrease as $1/\omega^2$. At the resonant frequency, $\omega=\omega_0$ the response goes to infinity. This causes the motion at this frequency to grow without limit.

With damping, the response has a similar structure albeit with different amplitudes. Namely at the resonance the response does not approach infinity but instead $R(\omega)=-i Q$. The quality factor, hence the amount of damping, limits the maximum resonant motion for a given input torque. 

\begin{figure}[!h]
\begin{centering}
\includegraphics[width=\textwidth]{ResponseFunction.pdf}
\caption{The response function for a simple torsion balance system with a resonance of $\omega_0=2\pi\ (0.1 \text{ Hz})$.}\label{respPlot}
\end{centering}
\end{figure}

The full response function is plotted in Figure \ref{respPlot} for a variety of quality factors and a resonant frequency of 0.1 Hz. As can be seen, below the resonance the magnitude of the response approaches unity with a nearly zero phase for most values of $Q$. There's a peak at the resonant frequency whose amplitude is strongly dependent on $Q$ and the phase undergoes a rapid transition. Above the resonance the magnitude follows $\sim1/\omega^2$ with a phase of $180^\circ$ independent of $Q$-value. Note that a phase of $180^\circ$ is equivalent to a negative value.


\chapter{Mechanics}
\section{Torque Sensing}

\quad For many experiments, the primary use of a torsion balance is to sense weak torques acting on the pendulum. In this mode, the measurements of the angle of the pendulum is converted to torque by rearranging Equation \ref{four2}:
\begin{equation}
\tau_{\text{ext}} (\omega)= \frac{\kappa\ \tilde{\theta}(\omega)}{R(\omega)} \label{torq}
\end{equation}

If we assume a frequency independent angle spectrum, $\tilde{\theta}(\omega)$, (which is unrealistic but a rough approximation of spectra arising from readout noise, Section \ref{readout}) then the corresponding torque spectrum will follow the inverse of the pendulum response function, $R(\omega)$. An example of such a spectrum is shown in Figure \ref{torqSpec}. This spectrum has very similar features at the response discussed in Section \ref{resp}. However, instead of having a peak at the resonant frequency it has a dip and above the resonant frequency the torque spectrum rises as $\sim \omega^2$. 

These features are the first example we've seen that expresses the significance of the frequency of signal of interest in designing an experiment. A careful experimenter would design an apparatus to minimize the noise at the frequency of interest within practical limits. For example, the model apparatus described by Figure \ref{torqSpec} would not be ideal to run an experiment with a signal at 1 Hz but would be well suited for a signal at 10 mHz since the noise level is differs by a factor of 100 between these two frequencies. This is a common consideration in torsion balance design that will reoccur throughout this work.

\begin{figure}[!h]
\begin{centering}
\includegraphics[width=\textwidth]{TorqueSpectrum.pdf}
\caption{Example torque spectrum assuming a frequency independent angle spectrum $\tilde{\theta}(\omega) = 1\ \text{nrad}/\sqrt{\text{Hz}}$ amplitude, $\kappa=10^{-8}\ \text{N m/rad}$, $Q=10$, and $\omega_0=2\pi\ (0.1 \text{ Hz})$ .}\label{torqSpec}
\end{centering}
\end{figure}
\pagebreak
\section{Inertial Sensing}

\quad A relatively newer use of torsion balances is for inertial sensing. Where as in the previous section we discussed torques acting on the pendulum, for inertial sensing the pendulum acts as an inertial proof mass. The goal of this mode is to measure the angular motion of structure that the pendulum is suspended from. These sorts of measurements have a wide range of application, from rotational seismology and seismic isolation to guidance systems and navigation. 

A system whose support structure is allowed to move can be modeled with Equation \ref{eom} by allowing $\theta_0$ to vary in time:

\begin{equation}
I~\ddot{\theta}(t)+\kappa(1+i\delta)  \big(\theta(t)-\theta_0(t)\big) = \tau_{\text{ext}}(t) \label{inert}
\end{equation}

Since the motion of the support is what we want to measure, Equation \ref{inert} can be transformed in the Fourier domain and rearranged to yield:

\begin{equation}
\tilde{\theta}(\omega)=\frac{1+i/Q}{1-\omega^2/\omega_0^2+i/Q}\ \tilde{\theta_0}(\omega) \label{inert2}
\end{equation}

However, angular readout systems do not sense the inertial angle of the pendulum but instead measure the difference in angle between the support and the pendulum. (In all other sections we assume the support is inertial.) Thus, the measured angle is:

\begin{equation}
\tilde{\theta_a}(\omega)=\tilde{\theta}(\omega)-\tilde{\theta}_0(\omega) \label{inert3}
\end{equation}

Combining Equations \ref{inert2} and \ref{inert3} yields:

\begin{equation}
\tilde{\theta}_0(\omega)=\frac{\omega_0^2}{\omega^2}\ \frac{1}{R(\omega)}\ \tilde{\theta}_a(\omega) \label{inert4}
\end{equation}

At first glance, Equation \ref{inert4} looks very similar to the angle response to torque, Equation \ref{four2}. However, the extra factor of $\omega^2/\omega_0^2$ mirrors the features onto opposite sides of the resonance. For a frequency independent measured angle spectrum, the inertial angle spectrum will follow $\sim1/\omega^2$ below the resonance while above it flattens to approach the measured angle spectrum. Figure \ref{inertSpec} shows an example of such a spectrum which displays the same features as Figure \ref{torqSpec} but mirrored about the resonance frequency.

\begin{figure}[!h]
\begin{centering}
\includegraphics[width=\textwidth]{InertialSpectrum.pdf}
\caption{Example inertial angle spectrum assuming a frequency independent measured angle spectrum $\tilde{\theta_a}(\omega) = 1\ \text{nrad}/\sqrt{\text{Hz}}$ amplitude, $Q=10$, and $\omega_0=2\pi\ (0.1 \text{ Hz})$.}\label{inertSpec}
\end{centering}
\end{figure}

Due to the features of Equation \ref{inert4}, an inertial sensing apparatus achieves its best performance above the resonant frequency and quickly looses sensitively below it. Thus lowering the resonant frequency increases the band of interest.

\section{Fiber Selection} \label{fiber}

\quad Although it is easy to think of the torsion fiber as a minor part of the torsion balance system, manufacturing the optimal fiber for a given experiment is an art in itself. There are two primary optimizations when designing a torsion fiber: minimizing the $\kappa$ of the fiber and maximizing the $Q$. These are not necessarily orthogonal and many design alterations will drastically change both parameters. 

There is generally three parameters that can be adjusted in a fiber design: the length of the fiber, $l$, the cross-sectional radius\footnote{Here we're assuming a circular cross-section but many fibers have more complex geometries which must be accounted for.}, $r$, and the material. To optimize these parameters, one usually uses the following procedure. First, find out the maximum length that can fit in the given apparatus and use that length of fiber. Then choose a material based on the requirements of the experiment, what is readily available to purchase or manufacture, and which material will give the highest $Q$. Then minimize the fiber radius until the weight of the pendulum is close to the breaking strength of the fiber (with a healthy safety factor to minimize break-ability). This can iterated through multiple times along with various apparatus changes (chamber height, charge mitigation, etc.) to achieve a near optimal set-up. 

Experimenter's are not left to blindly search for optimal parameters but are instead informed by vast troves of theoretical and experimental information much of which was produced by the engineering communities. The $\kappa$-value for a fiber with uniform cross-section can be readily calculated using \cite{hibbeler2003mechanics}:
\begin{equation}
\kappa = \mu \frac{J}{l} 
\end{equation}
where $\mu$ is the shear modulus of the material, $J$ is the torsional constant of the material, and $l$ is the length of the fiber. The shear modulus depends on the material and temperature whereas the torsional constant only depends on the cross-sectional geometry of the fiber. Assuming a circular cross-section this becomes:
\begin{equation}
\kappa = \mu \frac{\pi r^4}{2 l} \label{kappa}
\end{equation}
where $r$ is the cross-sectional radius of the fiber. Equation \ref{kappa} shows the drastic dependence of $\kappa$ on the radius of the fiber and to a lesser extent the length. The value of the shear modulus, $\mu$, for a given material is typically looked-up in various engineering references. Table \ref{ShearTable} gives the values for a selection of materials typically used in torsion balance apparatus.

\begin{center}		
	\begingroup
	\setlength{\tabcolsep}{10pt} % Default value: 6pt
	\renewcommand{\arraystretch}{1.5} % Default value: 1
	\begin{table}[ht!]
		\begin{center}
			\begin{tabular}{ |c|c| }
				\hline
				Material & Shear Modulus, $\mu$ (GPa) \\
				\hline
				Beryllium Copper (Be-Cu) & 48\\
				Aluminum (Al), 6061-T6 & 224\\
				Tungsten (W) & 161 \\ 
				Titanium (Ti) & 41\\
				Steel & 75\\
				Fused Silica (SiO$_2$) & 31 \\
				\hline
				
			\end{tabular}
			\caption{Table of the shear modulus for a selection of typical materials \cite{shear, quartz}.}\label{ShearTable}
		\end{center}
	\end{table}
	\endgroup
\end{center}

Naively, Equation \ref{kappa} could be interpreted as allowing arbitrarily low $\kappa$-values since up to this point we have not set any limits on the minimum width of the torsion fiber. However, in addition to providing the restoring torque the fiber also has to hold the pendulum's weight. The maximum force that a fiber can experience without deforming\footnote{There are two tensile strengths one could use: the yield strength or the ultimate tensile strength. The yield strength is the point at which the material deforms while the ultimate tensile strength is when it breaks. The ultimate tensile strength is typically less than twice the yield strength. Here we use the yield strength as it is lower than the ultimate tensile strength and deformation of a torsion fiber would alter the other parameters discussed here.} follows:
\begin{equation}
F=\sigma \pi r^2
\end{equation}
where $\sigma$ is the yield strength of the material and $r$ is the cross-sectional radius. The minimum radius needed to hold a pendulum is then:
\begin{equation}
r_{\text{min}}=\sqrt{\frac{m g}{\pi \sigma}}
\end{equation}
where $m$ is the mass of the pendulum and $g$ is the local gravitational acceleration. A fiber of this radius could hold the pendulum but any further force (earthquake, anthropogenic forces, etc.) on the fiber would cause it to yield. Thus a safety factor is usually used to ensure the apparatus is robust against laboratory conditions.
\begin{equation}
r_{\text{safe}}=\eta \sqrt{\frac{m g}{\pi \sigma}}
\end{equation}
where $\eta$ is the safety factor with typical values in the range of 2 - 3. Yield strength is another parameter which in practice is looked-up in engineering references. Table \ref{YieldTable} shows values for a collection of typical materials used for torsion fibers.

\begin{center}		
	\begingroup
	\setlength{\tabcolsep}{10pt} % Default value: 6pt
	\renewcommand{\arraystretch}{1.5} % Default value: 1
	\begin{table}[ht!]
		\begin{center}
			\begin{tabular}{ |c|c| }
				\hline
				Material & Yield Strength, $\sigma$ (MPa) \\
				\hline
				Beryllium Copper (Be-Cu) & 965 - 1205 \\
				Aluminum (Al), 6061-T6 & 241\\
				Tungsten (W) & 550 \\ 
				Titanium (Ti) & 100 - 225\\
				Steel & 75\\
				Fused Silica (SiO$_2$) & 48\\
				\hline
				
			\end{tabular}
			\caption{Table of yield strengths for a selection of typical materials \cite{quartz, yield, alum, becu}.}\label{YieldTable}
		\end{center}
	\end{table}
	\endgroup
\end{center}

The quality factor, $Q$, is also an important consideration when engineering a torsion fiber. The $Q$-value is inversely proportional to the amount of mechanical energy dissipated in the fiber. The $Q$-value not only changes the response function of the system, as can be seen in Figure \ref{respPlot}, but also influences the amount of thermal noise in the system. Thermal noise is discussed in detail in Section \ref{thermal} however when it comes to fiber design the important fact is that thermal torque noise follows, $\tau \sim 1/\sqrt{Q}$. Thus the higher the $Q$ the lower the thermal noise.

The exact value of the quality factor of a given fiber is dependent on a variety of factors, from the mechanics of the end points to the presence of surface defects, however many materials are known to have larger $Q$-values than others. Generally, if the material is more uniform then it will have a larger $Q$-value. Metals are the traditional choice especially those historically used in musical instruments (bells, guitar strings, etc.). Specifically, tungsten fibers are known to achieve $Q$-values of $\sim4500$ \cite{qual}. A type of glass called fused silica or fused quartz is also well known to be a high-$Q$ material. It is formed from almost pure silicon dioxide (SiO$_2$) and can achieve $Q$-values as high as $10^5$ \cite{qual}. As compared to metals, fused silica is extremely delicate and thus many hours of lab work can be ruined by simply touching the fiber. To many experimenter's, including those at the world's gravitational wave observatories, this risk is well worth the exceedingly high $Q$-values achieved by this material.

\section{Pendulum Design}

\chapter{Complications}
\section{Swing Modes} \label{swing}
\section{Centrifugal Force}
\section{Multiple Pendulums}

\chapter{Noise Sources and Mitigation}
\section{Thermal Noise} \label{thermal}
\section{Readout Noise}\label{readout}
\section{Seismic Motion}
\section{Electrostatic Couplings}
\section{Magnetic Noise}
\section{Gas Damping} \label{gas}
\section{Gravity Gradients}

\chapter{Case Study Experiments}
\section{Inverse Square Law}
\section{Equivalence Principle}
\section{Gravitational Wave Detection}

\bibliographystyle{unsrt}
\bibliography{TorsionBalances}

\end{document}
